\documentclass[chaparabic,ee,ms,12pt,oneandhalf]{metu}
\usepackage{appendix}
\usepackage{longtable}
\usepackage[pdftex]{hyperref}
\usepackage[all]{hypcap}
\usepackage{todonotes}

\graphicspath{ {./figures/} }

\usepackage[figuresright]{rotating}
\usepackage{xy} 
\usepackage{booktabs}
\usepackage{pifont}
\usepackage{color}
\usepackage{listings}
\usepackage{pdfpages}
\usepackage{array}
\usepackage{algorithm}
\usepackage{algorithmic}
\usepackage{float}
\usepackage{caption}
\usepackage{lastpage}
\usepackage{afterpage}
\usepackage{lipsum}
\usepackage{adjustbox}
\usepackage{rotating}


% \usepackage{graphicx}
\usepackage{amsmath,amssymb} % define this before the line numbering.
% \usepackage{ruler}
\usepackage{color}
% \usepackage{cite}
% \usepackage[utf8x]{inputenc}
% \usepackage{footnote}
% \makesavenoteenv{tabular}
% \makesavenoteenv{table}

\renewcommand{\sectionautorefname}{\S}
\renewcommand{\subsectionautorefname}{\S}

\newcommand{\norm}[1]{\left\lVert#1\right\rVert}

\captionsetup{belowskip=12pt,aboveskip=8pt}
\newcommand{\tab}{\hspace*{2em}}
\DeclareGraphicsExtensions{.pdf,.png,.jpg}


\usepackage{amsmath}
\usepackage{siunitx}
\usepackage{textcomp}
\usepackage{subcaption}


\usepackage{tikz}
\usepackage{mathtools}
% \usepackage{rotating}
%\PassOptionsToPackage{figuresright}{rotating}
\usetikzlibrary{shapes,arrows} % HIU added

\DeclarePairedDelimiter\ceil{\lceil}{\rceil}
\DeclarePairedDelimiter\floor{\lfloor}{\rfloor}

\DeclareMathOperator*{\argmax}{arg\,max}

\newcommand{\HIU}[1]{\textcolor{red}{[HIU: #1]}}
\newcommand{\BULLET}[1]{\textcolor{blue}{- #1}}
\newcommand{\hide}[1]{}
\newcommand{\etal}{{et al. }}
\pdfstringdefDisableCommands{\let\uppercase\relax}

\hyphenation{learn-ing}
\hyphenation{sec-tion}

% Name and Surname
\author{}
% Thesis Title English and Turkish
\title{}
\turkishtitle{}

\date{September 2020}
 
% prof : Prof. Dr.
% assocprof : Assoc. Prof. Dr.
% assistprof : Assist. Prof. Dr.
% dr : Dr.
%
% Director of Institute
\director[prof]{}
% Head of Department
\headofdept[prof]{}
%
% Supervisor : English and Turkish
\supervisor[assocprof]{}
% \turkishsupervisor{  } %if you will hard-code the academic title
%
% Affiliation of Supervisor in English and possibly in Turkish
\departmentofsupervisor{Electrical and Electronics Engineering, METU}

\cosupervisor[]{}
\departmentofcosupervisor{}
%
% Committee Members
% In general members are sorted according to their academic titles
%
% Proffesors (1)
% Associate Professors (2)
% Assistant Professors (3)
% Other (4)
% 
% IMPORTANT:  All affiliatons should fit in a single line
% If affiliation line is broken into two lines you should shorten the affiliation by using 
% abbrevations or any other means
%
% First committee member should be the chair of examining committee
% Typically the chair is one of the highest ranked committee members
% Ask your supervisor if you are not sure
\committeememberii[]{}
\affiliationii{Electrical and Electronics Engineering, METU}
% Second committee member is always your supervisor
\committeememberiii[]{}
\affiliationiii{Computer Engineering, METU}
% If you are an M.Sc. student and your Co-Supervisor is in your 
% examination committee, then third committee member is always your co-supervisor
%
% IMPORTANT: If you are Ph.D. student your co-supervisor can not be in your 
% examination committee.

% \def\@proftitlename{Prof. Dr.}\def\@tproftitlename{Prof. Dr.}
% \def\@assocproftitlename{Assoc. Prof. Dr.}\def\@tassocproftitlename{Doç. Dr.}
% \def\@assistproftitlename{Assist. Prof. Dr.}\def\@tassistproftitlename{Yrd. Doç. Dr.}
% \def\@drtitlename{Dr.}\def\@tdrtitlename{Dr.}

\committeememberiv[]{}
\affiliationiv{Electrical and Electronics Engineering, METU}
% Fourth committee member
\committeememberi[prof]{}
\affiliationi{Electrical and Electronics Engineering, METU}
% Fifth committee member
\committeememberv[]{}
\affiliationv{Computer Engineering, Ankara University}
%
% Keywords : English & Turkish, Comma seperated
\keywords{}
\anahtarklm{}
%
% Abstract in English
%
\abstract{


}
%
% Turkish Abstract
%
\oz{


} 
%
% Dedication 
\dedication{To my dear ...}
%
%
% Acknowledgements   
\acknowledgments{

}

%
% End of Personal and Introductory Information
%%%%%%%%%%%%%%%%%%%%%%%%%%%%%%%%%5
\begin{document}

% Preliminaries
\begin{preliminaries}
% If you are willing to use any custom stuff before Chapters, put it here
% Such as List of Abbreviations
% Check the abbreviations.tex for a template list of abbreviations

\begin{theglossary}{LONGESTABBRV}

\item[2D]   2 Dimensional
\item[3D]   3 Dimensional
\item[UAV]  Unmanned Aerial Vehicle
\item[MDP]  Markov Decision Processes
\item[RL]   Reinforcement Learning
\item[DRL]  Deep Reinforcement Learning
\item[TRPO] Trust Region Policy Optimization
\item[PPO]  Proximal Policy Optimization
\item[PSO]  Particle Swarm Optimization
\item[MLP]  Multilayer Perceptron
\item[KL]   Kullback–Leibler
\item[PD]   Proportional-Derivative
\item[CoM]  Center of Mass
\item[p2p]  Point-to-point

\end{theglossary}

\newpage

\begin{theglossaryvar}{LONGESTABBRV}

\item[$M$]  Mass of bi-rotor body
\item[$m$]  Mass of tail
\item[$I$]  Inertia of bi-rotor body
\item[$I_{tail}$]  Inertia of tail
\item[$g$]  Gravitational acceleration
\item[$\mathbf{x}$]  State vector of bi-rotor system
\item[$p_x$]  Position of bi-rotor in x-direction
\item[$p_y$]  Position of bi-rotor in y-direction
\item[$\vartheta$]  Orientation of bi-rotor
\item[$v_x$]  Velocity of bi-rotor in x-direction
\item[$v_y$]  Velocity of bi-rotor in y-direction
\item[$\omega$]  Angular velocity of bi-rotor
\item[$\varphi$]  Orientation of tail with respect to bi-rotor body
\item[$\mathbf{u}$]  Input vector of bi-rotor system
\item[$f_l$]    Thrust of left rotor
\item[$f_r$]    Thrust of right rotor
\item[$\tau_{tail}$]   Torque applied to tail
\item[$\tau$]   Equivalent torque of left and right rotor thrusts
\item[$f$]   Sum of left and right rotor thrusts
\item[$s$]   State of an MDP
\item[$a$]   Action of an MDP
\item[$r$]   Reward of an MDP


\end{theglossaryvar}

% End of Preliminaries
\end{preliminaries}
%   
% Latex content Goes Here 
% 
%

\setlength{\parindent}{0em}
\setlength{\parskip}{10pt}

% You can add as many chapters

\hide{
%Notes for yourself
}

\chapter{introduction}
\label{chp:intro}


\section{Motivation and Problem Definition}  %subject to change
\label{sec:intro_motivation}

%\begin{figure}[ht]
%   \centering
%    \includegraphics[width=0.6\textwidth]{figures/crazyflie.jpg}
%    \caption{A crazyflie 2.0 quad-rotor UAV \cite{bitcraze2016crazyflie}}
%    \label{fig:quad_sample_image}
%\end{figure}


\section{Contributions} %subject to change

\section{The Outline of the Thesis} %subject to change

\chapter{Related Work}%subject to change
\label{chp:relwork}

\section{Deep Reinforcement Learning}

\section{Curriculum Learning}

\section{Multi-rotor Control}



\chapter{Mathematical Background}
\label{chp:background}


\section{Bi-rotor Dynamics and Control}
\label{sec:dynamics}


\chapter{Conclusions}
\label{chp:conclusion}







\input{references.tex}

%
% References in Bibtex format goes into below indicated file with .bib extension
%\bibliography{thesis_references}
% You can use full name of authors, however most likely some of the Bibtex entries you will find, will use abbreviated first names
% If you don't want to correct each of them by hand, you can use abbreviated style for all of the references

%\bibliographystyle{abbrv}

% if you have more that one appendix, then use \appendices, otherwise use 
% \appendix
% \input{appendix/appendix1.tex}
\end{document}
